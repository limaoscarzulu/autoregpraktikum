\documentclass[11pt]{article}
\usepackage[utf8]{inputenc}
\usepackage{ngerman}
\usepackage{datetime}
\usepackage{amsmath}
\usepackage{xcolor}
\usepackage{geometry}

\renewcommand{\familydefault}{\sfdefault}

 \geometry{
 a4paper,
 total={170mm,257mm},
 left=20mm,
 top=20mm,
 }

\begin{document}
\begin{center}
\text{Praktikum Mess- und Regelungstechnik SoSe2022} \\ \textbf{\Large \color{blue} Cheatsheet ROS} \\ Simon Klüpfel, Lukas Zeller
\end{center}
\section*{Initialize and source workspace}
\begin{enumerate}
    \item terminal setup ros environment:
    \begin{verbatim}
        source /opt/ros/noetic/setup.bash
    \end{verbatim}
    \item catkin workspace im home erstellen
    \begin{verbatim}
        $ mkdir -p ~/catkin_ws/src
        $ cd ~/catkin_ws/
        $ catkin_make
    \end{verbatim}
    \item source this workspace
    \begin{verbatim}
        source devel/setup.bash
    \end{verbatim}
    \item check
    \begin{verbatim}
        $ echo $ROS_PACKAGE_PATH 
    \end{verbatim}
    \textit{should output:}
    \begin{verbatim}
        /home/youruser/catkin_ws/src:/opt/ros/noetic/share
    \end{verbatim}
\end{enumerate}
\section*{Commands}
\begin{enumerate}
    \item rospack \textit{allows you to get information
    about packages} \\
    \begin{verbatim}
        $ rospack find roscpp
    \end{verbatim}
    \item roscd \textit{ is part of the } rosbash \textit{suite. It allows you to change directory} \\
    \begin{verbatim}
        $ roscd roscpp/subdir
    \end{verbatim}
    \item pwd \textit{see current directory}
    \item rosls \textit{allows you to directly ls in a package by name rather than by absolute path}

    \item Creating packages
    \begin{verbatim}
        $ cd yourcatkinworkspace/src
        $ catkin_create_pkg packagename pkgdependency1 pkgdependency2
    \end{verbatim}
    \item Building packages/rebuild workspace
    \begin{verbatim}
        $ cd yourcatkinworkspace/src
        $ catkin make
        ## source workspace first! ##
        package.xml file provides meta inforamtion about the package
    \end{verbatim}
    \item Starting roscore (\textit{Terminal is then occupied!})
    \begin{verbatim}
        $ roscore
    \end{verbatim}
    \item Display information about the ROS nodes that are currently running
    \begin{verbatim}
        $ rosnode list
    \end{verbatim}
    \textit{should output:}
    \begin{verbatim}
        rosout ....
    \end{verbatim}
\end{enumerate} 
\section*{Dictionary}
\begin{enumerate}
    \item Nodes \textit{is an executable that uses ROS to communicate with other nodes} 
    \item Messages \textit{ROS data type used when subscribing or publishing to a topic}
    \item Topics \textit{Nodes can publish messages to a topic as well as subscribe to a topic to receive messages}
    \item Master \textit{Name service for ROS (i.e. helps nodes to find each other)}
    \item rosout \textit{ROS equivalent for stdout/stderr}
    \item roscore \textit{Master + rosout + parameter server (parameter server will be introduced later)}
\end{enumerate}
\section*{Good to know}
\begin{enumerate}
    \item LAN-Config: (\textit{in Ubuntu Network Settings})\\
    Address: 192.168.0.55, Netmask: 255.255.255.0
    \item Append AMCL Parameters:
    \textit{ go to:} \begin{verbatim} amcl_diff_cfg.yaml location: ~/catkin_ws/src/volksbot/launch/config \end{verbatim}
    \item Origin in RViz:
    \textit{ [-42.400000, -20.000000, 0.000000]}
\end{enumerate}
\end{document}