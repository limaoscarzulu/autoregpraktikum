\documentclass[11pt,a4paper]{article}
\usepackage{amsmath, graphicx, geometry}
\usepackage{hyperref, caption, subcaption}


\usepackage{typearea}
\areaset{160mm}{250mm}

\newcommand{\vcenteredinclude}[1]{\begingroup
\setbox0=\hbox{\includegraphics{#1}}%
\parbox{\wd0}{\box0}\endgroup}

%% better: (general command to vertically center horizontal material)
\newcommand*{\vcenteredhbox}[1]{\begingroup
\setbox0=\hbox{#1}\parbox{\wd0}{\box0}\endgroup}

\begin{document}
\title{\Large\bf Arbeit zum \\ \textit{Praktikum Mess- und Regelungstechnik} \\ Sommersemester 2022}
\author{Simon Klüpfel, Lukas Zeller \\
  Robotik und Telematik \\
  Universität Würzburg\\
  Am Hubland, D-97074 Würzburg\\
\small \texttt{lukas.zeller@stud.uni-wuerzburg.de} \\
\small \texttt{simon.kluepfel@stud.uni-wuerzburg.de}}
\date{Würzburg, 24.08.2022}

\maketitle

\section{Einleitung und Aufsetzen des Volksbot-Roboters}
In dieser Arbeit befassen wir uns mit dem Robot Operating System, kurz \textit{ROS}, und der 
Verwendung dessen auf einem einfachen Roboter, dem \textit{Volksbot}. \\
Der verwendete Roboter ist der \textit{RT3-2} mit zwei passiven Rädern und der \textit{RT-3} mit einem passiven Rad.
Entwickelt wurden diese vom \textit{Fraunhofer Institut IAIS} aus Sankt Augustin. \\
Die technischen Daten sind wie folgt: \\
\vspace{-5mm}
\begin{center}
\begin{tabular}{| p{5cm} p{5cm} |}
  \hline
  Abmessungen & 580x520x315mm (L x B x H) \\
  Gewicht & 17kg \\
  
  Raddurchmesser & 260x85mm (aktive Räder) \\
   & 200mm (passive Räder) \\
  Maximale Geschwindigkeit & 2,2 $\frac{m}{s}$ \\
  
  Maximale Zuladung & 25kg \\
  \hline
\end{tabular} \\
\small{ Auszug aus \url{https://www.volksbot.de/rt3-de.php}}
\end{center}
Auf dem Roboter ist ein Laserscanner sowie Hardware zur Verbindung mit dem verwendeten
Laptop installiert. Außerdem lässt sich der Roboter über einen Joystick, ähnlich eines Gamecontrollers, manuell steuern. 
Wir verwenden sowohl vorgegebene ROS-Nodes, die uns von Institut für Robotik und Telematik zur Verfügung gestellt wurden, als auch angepasste Nodes die wir selbst erstellt beziehungsweise
geändert haben. 

\newpage

\section{GMapping mit dem Volksbot}
Als erstes erstellen wir nach dem Vertrautmachen mit dem Roboter und der Entwicklunsumgebung eine Karte des Informatikgebäudes. 
Dies geschieht mit dem Onboard-Laserscanner und dem vorgegebenen GMapping-Algorithmus.

KOM: das ist falsch. zuerst fahren wir daten in eine Bag und machen aus der bag ne map

todo: plots einfügen, kommentieren

KOM: wir brauchen dazu keine plots??? Wir haben nicht mal relevante. Ein bild der karte die wir rausbekommen haben reicht

\section{AMCL-Lokalisierung}
todo: vergleich AMCL-Odometrie

\section{Pfadverfolgung}
Pfade vergleichen, was gut, was schlecht, eventuell Code-Kommentare 

KOM: du musst auf die Koordinatensysteme von AMCL/Odom eingehen. Das ist das Hauptthema im Code.

\section{Auswertung}
todo: 
Testen Sie den Regler mit dem aufgenommenen Pfad aus dem Informatikgebäude. Vergleichen Sie das Verhalten, wenn einmal die Odometrie und einmal die von AMCL bestimmten
Positionen als Messwerte für die Regelung verwendet werden.

also: plots, kommentare, was ist besser, was schlechter?

KOM: An Plots brauchen wir hier Simulation, den rohen Pfad, AMCL, Odom, und Bilder was er IRL jeweils getan hat. Die Bilder kann ich noch machen.
\end{document}

