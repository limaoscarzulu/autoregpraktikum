\documentclass[11pt,a4paper]{article}
\usepackage{amsmath, graphicx, geometry}
\usepackage{hyperref, caption, subcaption}


\usepackage{typearea}
\areaset{160mm}{250mm}

\newcommand{\vcenteredinclude}[1]{\begingroup
\setbox0=\hbox{\includegraphics{#1}}%
\parbox{\wd0}{\box0}\endgroup}

%% better: (general command to vertically center horizontal material)
\newcommand*{\vcenteredhbox}[1]{\begingroup
\setbox0=\hbox{#1}\parbox{\wd0}{\box0}\endgroup}

\begin{document}
\title{\Large\bf Arbeit zum \\ \textit{Praktikum Mess- und Regelungstechnik} \\ Sommersemester 2022}
\author{Simon Klüpfel, Lukas Zeller \\
  Robotik und Telematik \\
  Universität Würzburg\\
  Am Hubland, D-97074 Würzburg\\
\small \texttt{lukas.zeller@stud.uni-wuerzburg.de} \\
\small \texttt{simon.kluepfel@stud.uni-wuerzburg.de}}
\date{Würzburg, 24.08.2022}

\maketitle

\section{Einleitung}
In dieser Arbeit befassen wir uns mit dem Robot Operating System, kurz \textit{ROS}, und der 
Verwendung dessen auf einem einfachen Roboter, dem \textit{Volksbot}. \\
Der verwendete Roboter ist der \textit{RT3-2} mit zwei passiven Rädern und der \textit{RT-3} mit einem passiven Rad.
Entwickelt wurden diese vom \textit{Fraunhofer Institut IAIS} aus Sankt Augustin. \\
Die technischen Daten sind wie folgt: \\
\vspace{-5mm}
\begin{center}
\begin{tabular}{| p{5cm} p{5cm} |}
  \hline
  Abmessungen & 580x520x315mm (L x B x H) \\
  Gewicht & 17kg \\
  
  Raddurchmesser & 260x85mm (aktive Räder) \\
   & 200mm (passive Räder) \\
  Maximale Geschwindigkeit & 2,2 $\frac{m}{s}$ \\
  
  Maximale Zuladung & 25kg \\
  \hline
\end{tabular} \\
\small{ Auszug aus \url{https://www.volksbot.de/rt3-de.php}}
\end{center}
\begin{figure}
\centering
\begin{minipage}{.5\textwidth}
  \centering
  \includegraphics[width=.4\linewidth]{f710small.png}
  \captionof{figure}{Logitech F710 Joystick}
  \label{fig:test1}
\end{minipage}
\begin{minipage}{.5\textwidth}
  \centering
  \includegraphics[width=.4\linewidth]{rt3small.png}
  \captionof{figure}{RT3 Volksbot}
  \label{fig:test2}
\end{minipage}
\end{figure}


Auf dem Roboter ist ein Laserscanner sowie Hardware zur Verbindung mit dem verwendeten
Laptop installiert. Außerdem lässt sich der Roboter über einen Joystick, ähnlich eines Gamecontrollers, manuell steuern.\\
\newpage

\section{Weitere Abschnitte}
To do.

\section{Zusammenfassung und Ausblick}


\end{document}

